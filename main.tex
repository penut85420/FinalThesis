\documentclass[12pt]{article}

\usepackage{multirow}
% \usepackage{xeCJK}
\usepackage{amsmath}
\usepackage{graphicx}
\usepackage{subcaption}
\usepackage{setspace}
\usepackage[backend=biber,style=verbose-trad2]{biblatex}
\usepackage{wallpaper}
\usepackage[margin=2.8cm]{geometry}
\addtolength{\wpXoffset}{+9.26cm}
\addtolength{\wpYoffset}{-12.5cm}
\bibliography{hello}

\usepackage{booktabs}
\usepackage{siunitx} % Required for alignment
\sisetup{
  round-mode = places, % Rounds numbers
  round-precision = 2, % to 2 places
}
\usepackage{listings}
\usepackage{color}

\title{Hello This is Title with LaTeX}
\author{Wei-Ting Chen}

\begin{document}
\maketitle
\pagenumbering{gobble}
\newpage
\CenterWallPaper{0.1}{ntou.png}
\doublespacing
\tableofcontents
\singlespacing
\newpage

\section{Introduction}

\subsection{Motivation}

\subsection{Related Research}

\subsection{Paper Architecture}

\section{Datasets}
\paragraph{}
We use several traditional Chinese QA datasets such as NTCIR-CLQA, FGC, and DRCD. Each of these dataset contains many DQA pairs.

\subsection{NTCIR-CLQA}
\paragraph{}
NTCIR-CLQA include NTCIR-5 CLQA and NTCIR-6 CLQA. We mixed two datasets and retrieved only traditional Chinese part of datasets.

\subsection{FGC}
\paragraph{}
The FGC (Formosa Grand Challenge) QA datasets includes 1,271 questions with 150 paragraphs.

\subsection{DRCD}
\paragraph{}
The DRCD (Delta Reading Comprehension Dataset) is an open domain traditional Chinese MRC dataset. The dataset contains 2,108 articles from Wikipedia and 10,014 paragraphs with over 30,000 questions.

\section{Methods}
\paragraph{}
In this paper, we focus on how deep learning methods work on traditional Chinese QA datasets.

\section{Experiments}

\section{Conclusion}

\end{document}
