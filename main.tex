% !TeX encoding = UTF-8
\documentclass{article}

\usepackage{CJKutf8}
\usepackage{multirow}
\usepackage{amsmath}
\usepackage{graphicx}
\usepackage{subcaption}
\usepackage{setspace}
\usepackage{wallpaper}
\usepackage[hyphens]{url}
\usepackage[margin=2.8cm]{geometry}
\addtolength{\wpXoffset}{+9.26cm}
\addtolength{\wpYoffset}{-12.5cm}

\usepackage{booktabs}
\usepackage{siunitx} % Required for alignment
\sisetup{
  round-mode = places, % Rounds numbers
  round-precision = 2, % to 2 places
}
\usepackage{listings}
\usepackage{color}

\lstset{
  language=XML,
  morekeywords={encoding,
    xs:schema,xs:element,xs:complexType,xs:sequence,xs:attribute}
}

\title{RITE}
\author{
  Wei-Ting Chen\\
  National Taiwan Ocean University\\
  \texttt{10757025@mail.ntou.edu.tw}\\
  \\
  Chuan-Jie Lin\\
  Nation Taiwan Ocean University\\
  \texttt{cjlin@mail.ntou.edu.tw}\\
}

\begin{document}

\maketitle
\pagenumbering{gobble}

\newpage

\CenterWallPaper{0.1}{ntou.png}
\doublespacing
\tableofcontents
\singlespacing

\newpage

\pagenumbering{arabic}
\begin{CJK*}{UTF8}{bsmi}
\section{Introduction}

\subsection{Motivation}
\paragraph{}
Here is the motivation of this research.

\subsection{Related Work}
\paragraph{}
Here is the related work.

\subsection{NLI Dataset}
\paragraph{}
Related NLI dataset, SNLI, MNLI, QNLI, CNLI, OCNLI, RITE.

\subsection{Thesis Architecture}
\paragraph{}
Here is the architecture of this thesis.

\section{Datasets}
\paragraph{}
% NLI 資料集是數組句對的集合,每組句對由一個「前提句」與一個「假設句」所組成,這兩句話會形成一個推論,用來表示兩句話之間的關係,這樣的任務被稱為 NLI / RTE。而本論文主要研究的對象 - RITE 資料集則是使用了雙向、單向、互斥與獨立四個關係。
An NLI dataset is a collection of sentence pairs, each sentence pairs, premise and hypothesis, has an inference, which indicates the relation of two sentences, this kind of task is known as natural language inference (NLI), also known as recognizing textual entailment (RTE).

\paragraph{}
% 常見的 NLI 資料集例如 SNLI, MNLI 與 CNLI 等,通常使用了蘊含、互斥與獨立三個標籤
The relations of common NLI datasets, for example SNLI\cite{snli:emnlp2015}, MNLI\cite{N18-1101}, CNLI\footnote{\label{cnli}https://github.com/blcunlp/CNLI}, and OCNLI\cite{ocnli}, usually represent in three types of labels: entailment, contradiction, and neural.

\subparagraph{entailment} means that the premise entails the hypothesis, and whether the hypothesis entails the premise does not matter.
\subparagraph{contradiction} means that the premise and the hypothesis cannot be true at the same time.
\subparagraph{neural} means that the premise does not have any relation mentioned above with the hypothesis.

\paragraph{}
And in this thesis, our main target datasets - RITE, have four types of labels: bi-directional entailment, forward entailment, contradiction, and independence.

\subparagraph{bi-directional entailment} means that the premise entails the hypothesis, and the hypothesis entails the premise.
\subparagraph{forward entailment} means that the premise entails the hypothesis, but the hypothesis does not entail the premise.
\subparagraph{independence} as same as the meaning of the label neural.

\subsection{RITE}
\paragraph{}
RITE (Recognizing Inference in TExt) is the sub task in NTCIR conference, including Traditional Chinese, Simplified Chinese, and Japanese. In this research, we use the Traditional Chinese of RITE2 and RITE-VAL as our major evaluation datasets, which are come from NTCIR-10\cite{ntcir10rite2} and NTCIR-11\cite{ntcir11rite-val} these two conferences separately. They are collected from variety topics, such as domestic, history, politics, medicine and economy.

% 在 NTCIR-9 的時候同樣有舉辦 RITE 的 Task,其資料集為 RITE1,然而他的 dev set 與 test set 都被放入 RITE2 的 dev set 裡面,所以本研究只有使用 RITE2。
\paragraph{}
There is a RITE task in NTCIR-9 too, and the dataset it used is called RITE1\cite{ntcir9rite1}. However, the dev set and test set of RITE1 had been merged into the dev set of RITE2, so we only use RITE2 in this research.

\paragraph{}
% RITE-VAL 與 RITE2 的格式基本上相同,他們都有前提句 t1 與假設句 t2 以及他們的 id 與 label,而 RITE-VAL 則額外提供了「種類」的資訊,用來表示他們的句對其推論關係所涉及到的語言現象。
The formatting of RITE-VAL and RITE2 are the same, they both have the premise $t_1$, the hypothesis $t_2$, their pair id, and the label, while RITE-VAL provides the information of "category" to indicates the linguistic phenomenon of the relation of the sentence pair. There are 28 linguistic phenomenon are defined in RITE-VAL.

\lstset{
  extendedchars=false,
  basicstyle=\ttfamily,
  keywordstyle=\color{blue},
  stringstyle=\color{purple},
  frame=lines,
  breaklines=true,
  showstringspaces=false,
  escapechar=\#,
}
% RITE2
\begin{lstlisting}[language=XML, caption=Example of RITE2]
<pair id="430" label="F">
  <t1>#長期使用類固醇會導致情緒不穩,幻覺和妄想症#</t1>
  <t2>#類固醇可能造成幻覺妄想症#</t2>
</pair>
\end{lstlisting}

% RITE-VAL
\begin{lstlisting}[language=XML, caption=Example of RITE-VAL]
  <pair id="48" label="C" category="antonym">
    <t1>#1927年末,蘇聯穀物短缺,史達林力求迅速消滅富農階級,並始推農業集體化政策。#</t1>
    <t2>#1927年末,蘇聯穀物過剩,史達林力求迅速消滅富農階級,並始推農業集體化政策。#</t2>
  </pair>
\end{lstlisting}

\subsection{MNLI}
% MNLI 是一個透過 Crowd-sourced 建立的資料集,他蒐集了來自多種文體的句子,包含口說與書寫的文字。
MNLI (Multi-Genre Natural Language Inference, MultiNLI) is a crowd-sourced dataset that collected sentences from a wide range of genres, both in spoken and written text.

\begin{minipage}{\linewidth}
\begin{lstlisting}[language=Python, caption=Example of MNLI]
{
  "annotator_labels": [
      "neutral",
      "entailment",
      "neutral",
      "neutral",
      "neutral"
  ],
  "genre": "slate",
  "gold_label": "neutral",
  "pairID": "63735n",
  "promptID": "63735",
  "sentence1": "The new rights are nice enough",
  "sentence1_binary_parse": "( ( The ( new rights ) ) ( are ( nice enough ) ) )",
  "sentence1_parse": "(ROOT (S (NP (DT The) (JJ new) (NNS rights)) (VP (VBP are) (ADJP (JJ nice) (RB enough)))))",
  "sentence2": "Everyone really likes the newest benefits ",
  "sentence2_binary_parse": "( Everyone ( really ( likes ( the ( newest benefits ) ) ) ) )",
  "sentence2_parse": "(ROOT (S (NP (NN Everyone)) (VP (ADVP (RB really)) (VBZ likes) (NP (DT the) (JJS newest) (NNS benefits)))))"
}
\end{lstlisting}
\end{minipage}

\subsection{CNLI}
\paragraph{}
CNLI (Chinese Natural Language Inference) is a Simplified Chinese dataset from a sub task of The Seventeenth China National Conference on Computational Linguistics (CCL 2018)\footnote{\label{foo}http://www.cips-cl.org/static/CCL2018/index.html}.

\begin{CJK*}{UTF8}{gbsn}
\begin{lstlisting}[language=Python, escapechar=\#, caption=Example of CNLI]
{
  "pid": "AE5175",
  "t1": "#\color{purple}穿红衬衫的男人和拿着白色袋子的女人正在交谈。#",
  "t2": "#\color{purple}两个人在交谈#",
  "label": "entailment"
}
\end{lstlisting}
\end{CJK*}

\subsection{OCNLI}
\paragraph{}
OCNLI (Original Chinese Natural Language Inference) is also a large-scale Simplified Chinese NLI dataset that does not rely on automatic translation or non-expert annotation.

\begin{minipage}{\linewidth}
\begin{CJK*}{UTF8}{gbsn}
\begin{lstlisting}[language=Python, escapechar=\#, caption=Example of OCNLI]
{
  "level": "medium",
  "sentence1": "#\color{purple}经济社会发展既有量的扩大,又有质的提升,为今后奠定了基础#",
  "sentence2": "#\color{purple}经济社会始终在向好的方向发展#",
  "label": "neutral",
  "label0": null,
  "label1": null,
  "label2": null,
  "label3": null,
  "label4": null,
  "genre": "gov",
  "prem_id": "gov_96",
  "id": 50434
}
\end{lstlisting}
\end{CJK*}
\end{minipage}

\section{Methods}

\subsection{SVM}

\subsection{Simple DNN}

\subsection{RA Model}

\subsection{BERT}

\section{Conclusion}
\paragraph{}
Here is the conclusion.

\bibliography{main}
\bibliographystyle{ieeetr}

\end{CJK*}
\end{document}

% Reference 要附上期刊集數頁數等 (journal, volume, pages)
