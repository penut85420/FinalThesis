\documentclass{article}

\usepackage{multirow}
\usepackage{xeCJK}
\usepackage{amsmath}
\usepackage{graphicx}
\usepackage{subcaption}
\usepackage{setspace}
\usepackage[backend=biber,style=verbose-trad2]{biblatex}
\usepackage{wallpaper}
\usepackage[margin=2.8cm]{geometry}
\addtolength{\wpXoffset}{+9.26cm}
\addtolength{\wpYoffset}{-12.5cm}
\bibliography{hello}
% \setCJKmainfont{UMing}
\usepackage{booktabs}
\usepackage{siunitx} % Required for alignment
\sisetup{
  round-mode = places, % Rounds numbers
  round-precision = 2, % to 2 places
}
\usepackage{listings}
\usepackage{color}

\lstset{ % General setup for the package
	language=C,
	basicstyle=\small\sffamily,
	numbers=left,
 	numberstyle=\tiny,
	frame=tb,
	tabsize=4,
	columns=fixed,
	showstringspaces=false,
	showtabs=false,
	keepspaces,
	commentstyle=\color{red},
	keywordstyle=\color{blue}
}
\title{My First Document}
\date{2020-12-31}
\author{Penut Chen}

\begin{document}
\maketitle
\pagenumbering{gobble}

\newpage

\CenterWallPaper{0.1}{ntou.png}
\doublespacing
\tableofcontents
\singlespacing
\newpage

\pagenumbering{roman}
\section{Learn about Section}
Hello world!
\subsection{Subsection Title}
Structuring a document is easy!
\subsubsection{Subsubsection Title}
More text.
\paragraph{Paragraph}
Some more text.
\subparagraph{Subparagraph}
Even more text.

\section{Package}
\begin{equation*}
    f(x)=x^2
\end{equation*}
\subsection{Inline Math}
\paragraph{}
This formula $f(x)=x^2$ is an example.
\subsection{Align}
No align
\begin{equation*}
    1+2=3
\end{equation*}
\begin{equation*}
    1=3-2
\end{equation*}
Align
\begin{align*}
    1+2&=3\\
    1&=3-2
\end{align*}
Complex
\begin{align}
    f(x)&=x^2\\
    g(x)&=\frac{1}{x}\\
    FF(x)&=\int^a_b\frac{1}{3}x^3\\
    GeO(x)&=\frac{1}{\sqrt(x)}
\end{align}

\newpage

\pagenumbering{arabic}
\section{Matrix}
\begin{equation}
    \left[
    \begin{matrix}
    1&0\\
    0&1
    \end{matrix}
    \right]
\end{equation}
\subsection{Big Left and Right}
\begin{equation}
    \left(
    \frac{1}{\sqrt{2}}
    \right)
\end{equation}

\section{Insert Figures}
Insert a figure is easy.
\begin{figure}[h!]
    \centering
    \includegraphics[width=0.5\linewidth]{rem_ram.jpg}
    \caption{Rem and Ram}
    \label{fig:rem_ram}
\end{figure}

\subsection{Insert More Figures}
Insert two figures is easy too!
\begin{figure}[h!]
    \centering
    \begin{subfigure}[b]{0.2\linewidth}
        \includegraphics[width=\linewidth]{rem_ram.jpg}
        \caption{1}
    \end{subfigure}
    \begin{subfigure}[b]{0.2\linewidth}
        \includegraphics[width=\linewidth]{rem_ram.jpg}
        \caption{2}
    \end{subfigure}
    \caption{More Rem and Ram}
\end{figure}

\newpage
\subsection{Insert So Many Figures}
Even for so many figures!
\begin{figure}[h!]
    \centering
    \begin{subfigure}[b]{0.2\linewidth}
        \includegraphics[width=\linewidth]{rem_ram.jpg}
        \caption{1}
    \end{subfigure}
    \begin{subfigure}[b]{0.2\linewidth}
        \includegraphics[width=\linewidth]{rem_ram.jpg}
        \caption{2}
    \end{subfigure}
    \begin{subfigure}[b]{0.2\linewidth}
        \includegraphics[width=\linewidth]{rem_ram.jpg}
        \caption{3}
    \end{subfigure}
    \begin{subfigure}[b]{0.6\linewidth}
        \includegraphics[width=\linewidth]{rem_ram.jpg}
        \caption{4}
    \end{subfigure}
    \caption{More Rem and Ram}
\end{figure}
\paragraph{}
Figure \ref{fig:rem_ram} shows Rem and Ram. 

\section{Citation Related}
% Yeah a citation \cite{cui2019pretraining}.
% So many citations: \cite{sun2019ernie}\cite{yang2019data}
% \cite{Cui_2019}\cite{cheng2019symmetric}\cite{hsu2019zeroshot}.
This is the first cite\autocite[*]{cui2019pretraining}, 
this is the second cite\autocite[*]{hsu2019zeroshot}.
Here are many cites \autocite[*]{cheng2019symmetric} \autocite[*]{yang2019data}
\newpage

\section{How About Table}
\begin{table}[h!]
    \centering
    \caption{Caption at top}
    \begin{tabular}{c|S|S|S|S}
         \multirow{2}{*}{Lab Name} & \multicolumn{1}{c|}{\textbf{Lab1}}
              & \multicolumn{1}{c|}{\textbf{Lab2}}
              & \multicolumn{2}{c}{\textbf{Lab3}} \\
        & $\beta$ & $\gamma$ & $\omega$ & $\theta$ \\
        \toprule
        Target & 23.14 & 1.23 & 0.3 & 0.2 \\
        Value & 4.38 & 17.89 & 0.989 & 0.1 \\
        \midrule
        \multirow{2}{*}{Multi} & 999.8 & 0.1 & 123.4 & 567.8 \\
        & 0.123 & 2 & 369.8 & 0.039 \\
        \bottomrule
    \end{tabular}
    % \caption{Caption at bottom}
    \label{tab:my_label}
\end{table}

\section{Footnote}
\paragraph{}
This is some example text\footnote{\label{myfootnote}Hello this is footnote}.
% \begin{CJK*}{UTF8}{bkai}
\section{使用中文}
\paragraph{}
對於 LaTeX 是否能使用繁體中文,還是抱持著疑慮;在這部分顯然有相當多的方法\footnote{\label{cjk}https://www.overleaf.com/learn/latex/chinese},原來 LaTeX 的 Compiler 也有相當程度的影響。
% \end{CJK*}
\section{List}
\subsection{Unordered Lists}
    \begin{itemize}
        \item Banana
        \item Orange
        \item Watermelon
    \end{itemize}
\subsection{Ordered Lists}
    \begin{enumerate}
        \item Sword
        \item Blade
        \item Gun
    \end{enumerate}
\subsection{Nested Lists}
    \begin{itemize}
        \item Get
            \begin{enumerate}
                \item Load
                \item Process
                \item Close
            \end{enumerate}
        \item Put
        \item Done
    \end{itemize}
\newpage

\section{C Programming}
\paragraph{}
    Coding in latex!
    \begin{lstlisting}
    #include <stdio.h>
    #define N 10
    
    int main() {
        int arr[N] = {0};
        
        printf("Input: ");
        for (int i = 0; i < N; i++)
            scanf("%d", &arr[i]);
        
        printf("Done!\n");
    
        return 0;
    }
    \end{lstlisting}
    \subparagraph{}
        Or load from a file!
        \lstinputlisting{demo.c}

\newpage

\printbibliography
% \bibliographystyle{ieeetr}

\newpage

\section{Appendix}
\begin{appendix}
    \listoffigures
    \listoftables
\end{appendix}

\end{document}